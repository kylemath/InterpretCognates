\section{Background}

Our work draws on two largely separate literatures: the geometry of multilingual neural representations and the cognitive science of how multilinguals organize meaning.
We review each in turn, highlighting the specific hypotheses that motivate our experiments.

\subsection{Multilingual Representation Geometry}

A central question in multilingual NLP is whether shared encoder models learn language-neutral representations or merely co-locate language-specific subspaces.
\citet{pires2019} provided early evidence for the former, demonstrating that multilingual BERT \citep{devlin2019} supports zero-shot cross-lingual transfer on NER and POS tagging even between typologically distant languages, suggesting the emergence of shared syntactic abstractions.
Subsequent work has refined this picture considerably.

\citet{chang2022} decomposed the representation space of XLM-R \citep{conneau2020} into language-sensitive and language-neutral axes using a probe trained to predict language identity.
They found that removing the top language-sensitive principal components improves cross-lingual alignment on semantic tasks, implying that language identity is encoded in a low-dimensional subspace largely orthogonal to semantic content.
This finding motivates our conceptual store experiment, which isolates the language-neutral component by subtracting per-language centroids.

The geometry of multilingual encoders is complicated by anisotropy---the tendency of learned representations to cluster in a narrow cone rather than occupying the full available volume.
\citet{rajaee2022} showed that multilingual BERT embeddings are highly anisotropic and that this degrades cross-lingual similarity estimates.
\citet{mu2018} proposed All-but-the-Top, a post-processing method that removes the mean and top principal components from word embeddings to improve isotropy.
Our mean-centering approach can be viewed as a per-language variant of this correction, adapted to the multilingual setting where the dominant direction of anisotropy differs across languages.

At a finer grain, \citet{voita2019} demonstrated that individual attention heads in Transformer models specialize for distinct linguistic functions, including positional, syntactic, and rare-token tracking.
\citet{foroutan2022} extended this line of work to the multilingual case, identifying language-neutral sub-networks within multilingual Transformers that activate consistently across languages for equivalent inputs.
These findings suggest that universality is not merely a global property of the representation space but is also reflected in modular internal structure.

Taken together, this literature establishes that multilingual Transformers encode both language-specific and language-neutral information in geometrically separable subspaces.
Our experiments test whether this geometric separation extends to NLLB-200---a model trained explicitly for translation across 200 languages---and whether the language-neutral component exhibits structure predicted by cognitive science.

\subsection{Cognitive Science of Multilingual Representation}

The question of whether bilinguals and multilinguals maintain a shared conceptual store has been debated for decades.
The Revised Hierarchical Model \citep{kroll1994,kroll2010} posits that bilinguals access a common conceptual store through language-specific lexical representations, with direct concept--word connections strengthening with proficiency.
The BIA+ model \citep{dijkstra2002} further proposes that bilingual word recognition involves non-selective lexical access: encountering a word in one language automatically activates representations in the other, mediated by a shared semantic level.

Neuroimaging evidence supports the existence of language-independent conceptual representations.
\citet{correia2014} used representational similarity analysis on fMRI data to show that the anterior temporal lobe (ATL) encodes semantic category information identically across languages in bilingual speakers, providing direct neural evidence for a language-independent conceptual hub.
More recently, \citet{deniz2025} used voxelwise encoding models on several hours of naturalistic narrative fMRI data to show that Chinese--English bilinguals employ largely shared semantic brain representations across languages, but that these representations undergo systematic fine-grained \emph{shifts} between languages---shifts that modulate how concept categories are weighted rather than which brain regions are recruited.
This finding of shared-but-modulated representations provides striking neural support for the geometric picture we advance computationally: a language-neutral conceptual core with language-specific offsets superimposed.
\citet{thierry2007} demonstrated using event-related potentials that Chinese--English bilinguals unconsciously activate Chinese phonological representations when processing English words, implying automatic cross-linguistic co-activation at a sub-lexical level.
More recently, \citet{malikmoraleda2022} investigated the fronto-temporal language network across 45 languages spanning 12 language families and found that the same network activates for all languages with consistent left-lateralization and functional selectivity, suggesting a universal neural substrate for language processing.

Cross-linguistic universals provide a complementary perspective.
\citet{swadesh1952} identified a core vocabulary of basic concepts (body parts, kinship terms, natural phenomena) that resists borrowing and changes slowly across all known languages, motivating its use as a probe for universal semantic structure.
The ASJP database \citep{jaeger2018} quantifies genetic distances between languages using Swadesh-list cognates, providing the phylogenetic ground truth for our phylogenetic correlation analysis.
\citet{berlin1969} demonstrated that languages partition the color space in strikingly similar ways, following an implicational hierarchy of basic color terms---a finding we test in our color circle experiment.

The colexification literature bridges cognitive and computational perspectives.
When unrelated languages independently lexify two concepts with the same word form (e.g., ``arm'' and ``hand'' in many languages), this provides evidence for cognitive proximity between those concepts \citep{list2018}.
The CLICS\textsuperscript{3} database \citep{rzymski2020} aggregates colexification patterns across thousands of languages, enabling the statistical test in our colexification experiment: if NLLB-200 has learned cognitively plausible semantic structure, colexified pairs should be closer in embedding space.

Cross-lingual word embedding research has also engaged with these cognitive questions.
\citet{vulic2020} introduced a large-scale multilingual evaluation benchmark covering 12 typologically diverse languages, noting that bilingual lexicon induction implicitly assumes a degree of isomorphism between monolingual semantic spaces---an assumption closely related to the shared conceptual store hypothesis.
Our offset invariance experiment directly tests this assumption by measuring whether semantic difference vectors are preserved across languages.

The present work is, to our knowledge, the first to systematically test predictions from bilingual lexical organization theories against the internal representations of a massively multilingual translation model spanning \NumLanguages{} languages.
