\section{Methods}
\label{sec:methods}

\subsection{Model and Data}

We probe the internal representations of NLLB-200, a massively multilingual neural machine translation system comprising 600M parameters in its distilled variant \citep{nllbteam2022}.
NLLB-200 employs an encoder-decoder Transformer architecture \citep{vaswani2017} with a shared encoder across all 200 supported languages, making it a natural test bed for investigating whether cross-lingual semantic structure emerges from translation-oriented training alone.

As our lexical probe we adopt the Swadesh core vocabulary list \citep{swadesh1952}, a standard tool in historical linguistics designed to capture culturally stable, universally attested concepts such as kinship terms, body parts, natural phenomena, and basic actions.
We embed all \NumConcepts{} Swadesh items across \NumLanguages{} languages supported by NLLB-200 (excluding a small set of languages whose encoder embeddings are degenerate outliers in our extraction pipeline), yielding a concept-by-language embedding matrix that serves as the basis for all downstream analyses.

To obtain contextual embeddings rather than decontextualized token representations, we place each target word in a fixed carrier sentence of the form \textit{``I saw a \{word\} near the river''}, translated into each target language.
This choice is motivated by the observation that Transformer encoder representations are highly context-dependent \citep{devlin2019}: a bare word input would yield an embedding dominated by positional and start-of-sequence artifacts rather than lexical semantics.
The carrier sentence provides a minimal, semantically neutral context that activates the target word's lexical representation while minimizing confounds from sentential semantics.
We then extract the encoder hidden states corresponding only to the target word's subword tokens, discarding activations from the carrier context.
When a word is split into multiple subword tokens by the SentencePiece tokenizer, we mean-pool their activations to produce a single vector per concept--language pair.
Agglutinative and polysynthetic languages tend to produce longer subword sequences, as morphological markers (case, evidentiality, tense) are segmented into additional tokens whose mean-pooling may attenuate fine-grained lexical features.
We note that this carrier sentence is English-derived and imposes structural assumptions (SVO order, articles, spatial prepositions) that are not typologically universal; we assess this confound in Sections~\ref{sec:results} and~5.
To support reproducibility, we release the full analysis code and the derived experiment outputs (JSON summaries, figures, and macro tables) used to build the paper; the pipeline can also be rerun from model weights and the included corpora.

To assess the impact of the carrier sentence on our results, we additionally extract decontextualized embeddings by embedding each target word in isolation, without any surrounding context.
This bare-word baseline tests whether the convergence patterns we report in Section~\ref{sec:results} are driven by the carrier sentence's syntactic scaffold or by the target word's intrinsic lexical representation.

\subsection{Embedding Extraction and Correction}

Raw contextual embeddings from large language models are known to occupy a narrow cone in representation space, exhibiting low isotropy that can inflate cosine similarity scores and obscure meaningful geometric structure \citep{mu2018,rajaee2022}.
We extract mean-pooled encoder hidden states from the final Transformer layer and apply a two-stage correction procedure.

\begin{figure*}[htbp]
    \centering
    \includegraphics[width=\textwidth,height=0.88\textheight,keepaspectratio]{figures/fig_water_manifold.pdf}
    \caption{Embedding geometry for the concept ``water'' across 29 languages. (a) 3D PCA projection colored by language family shows tight clustering despite orthographic diversity. (b) Pairwise similarity heatmap reveals that same-family languages (e.g., Romance, Slavic) cluster, but cross-family similarity remains high ($>0.93$ for most pairs).}
    \label{fig:water_manifold}
  \end{figure*}
  
First, we perform All-But-The-Top (ABTT) isotropy correction \citep{mu2018}: we subtract the global mean embedding computed over all concept--language pairs, then project out the top $k=3$ principal components of the centered matrix.
This removes the dominant directions that encode frequency- and language-identity information rather than semantics, yielding a more isotropic embedding space in which cosine similarity more faithfully reflects semantic relatedness.
The choice of $k=3$ is validated by a sensitivity analysis across a range of $k$ values (Section~\ref{sec:results}), which confirms that the convergence ranking is robust to this hyperparameter.

Second, for analyses that require disentangling concept-level structure from language-level clustering, we apply per-language mean-centering: we subtract each language's centroid (its mean embedding across all \NumConcepts{} concepts) before computing PCA or pairwise distances.
This correction factors out the systematic offset that each language occupies in the shared space and exposes the residual conceptual geometry shared across languages.

\subsection{Experiments}

We design six complementary experiments that probe distinct facets of the multilingual representation geometry, moving from broad lexical convergence patterns to fine-grained relational structure.

\paragraph{Swadesh Convergence Ranking.}
For each of the \NumConcepts{} Swadesh concepts we compute the mean pairwise cosine similarity across all $\binom{\NumLanguages{}}{2}$ language pairs, producing a per-concept convergence score.
Ranking concepts by this score reveals which meanings are encoded most uniformly across languages and which exhibit the greatest cross-lingual dispersion.

\paragraph{Phylogenetic Correlation.}
We test whether the geometry of the embedding space recapitulates known genetic relationships among languages.
We construct a language-by-language embedding distance matrix by averaging concept-level cosine distances over all Swadesh items, and compare it to the ASJP phonetic distance matrix \citep{jaeger2018} using the Mantel test with \MantelPermutations{} permutations to assess statistical significance.

\paragraph{Colexification Proximity.}
Colexification---the phenomenon whereby a single word form covers multiple concepts---reflects deep semantic associations that recur across unrelated languages \citep{list2018}.
We test whether NLLB-200's representations internalize these associations by comparing the cosine similarity of concept pairs that are colexified in the CLICS\textsuperscript{3} database \citep{rzymski2020} to those that are not, using a Mann-Whitney $U$ test with Cohen's $d$ as the effect size measure.

\paragraph{Conceptual Store Metric.}
Inspired by neuroscientific evidence for language-independent conceptual representations \citep{correia2014}, we quantify the degree to which concepts cluster by meaning rather than by language.
We compute the ratio of mean between-concept cosine distance to mean within-concept cosine distance, both on raw embeddings and after per-language mean-centering, and report the improvement factor.

\paragraph{Color Circle.}
We project the cross-lingual centroids of the 11 basic color terms identified by \citet{berlin1969} into a two-dimensional PCA space.
If the model has learned perceptually grounded color semantics from translation data alone, the resulting arrangement should recover the warm--cool opposition and the circular topology observed in human color perception.

\paragraph{Offset Invariance.}
Following the analogy-based reasoning paradigm introduced by \citet{mikolov2013}, we examine whether semantic relationships are encoded as consistent vector offsets across languages.
For \OffsetNumPairs{} concept pairs (e.g., \textit{fire}--\textit{water}, \textit{sun}--\textit{moon}), we compute the per-language offset vector and measure its cosine similarity to the centroid offset averaged over all languages \citep{chang2022}.
High cross-lingual consistency indicates that the model represents relational meaning in a language-invariant manner.

We supplement these six experiments with a validation analysis that assesses the robustness of our embedding corrections.

\paragraph{Isotropy Correction Validation.}
To verify that our ABTT correction does not distort the convergence signal, we compare the full Swadesh ranking under raw and corrected embeddings.
We compute the Spearman rank correlation $\rho$ between the two orderings and visually inspect the top-20 concepts under each regime.
A high correlation indicates that isotropy correction preserves the relative ordering of concepts while rescaling absolute similarity values to a more interpretable range.

To disentangle orthographic from semantic contributions to embedding convergence, we additionally regress convergence scores against mean orthographic and phonological similarity of each concept's word forms across Latin-script languages, reporting $R^2$ alongside the Swadesh convergence ranking.
