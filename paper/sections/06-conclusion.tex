\section{Conclusion}

We have presented a comprehensive suite of experiments probing the encoder representations of NLLB-200 across \NumLanguages{} languages and \NumConcepts{} Swadesh-list concepts, revealing structural parallels between the geometry of neural machine translation and cognitive theories of multilingual lexical organization.
Pairwise embedding distances correlate significantly with phylogenetic distances ($\rho = \MantelRho{}$, $p = \MantelP{}$), colexified concept pairs are embedded more closely than non-colexified pairs ($d = \ColexCohenD{}$), mean-centering per language exposes a shared conceptual store with a $\ConceptualStoreImprovement{}\times$ improvement in concept separability, and semantic difference vectors are remarkably consistent across languages (mean cosine $\OffsetMeanConsistency{}$).
Complementary analyses of Swadesh stability rankings and universal color terms provide converging evidence that the model encodes cross-linguistically stable semantic structure.
Regression controls confirm that orthographic similarity explains only $R^2 = \DecompRsqOrtho{}$ of convergence variance, isotropy validation shows near-perfect rank preservation ($\rho = \IsotropySpearmanRho{}$), and per-family disaggregation of offset consistency (Figure~\ref{fig:offset}b) reveals that relational structure is preserved even across typologically distant language families.

These results bridge NLP interpretability and cognitive science by demonstrating that the internal geometry of a multilingual Transformer trained solely on parallel text exhibits properties predicted by the BIA+ model \citep{dijkstra2002}, the Revised Hierarchical Model \citep{kroll2010}, and neuroimaging studies of language-independent conceptual hubs \citep{correia2014}.

Several directions remain open.
Per-layer trajectory analysis could reveal how language-specific and language-universal information separate across the encoder stack, complementing the attention-head decomposition approach of \citet{voita2019}.
Cross-model comparisons with XLM-R \citep{conneau2020} and mBERT \citep{devlin2019} would test whether the geometric regularities we observe are architecture-specific or emerge broadly in multilingual pretraining.
Extending the concept inventory to larger Swadesh sets and integrating typological features from WALS would strengthen the link between embedding geometry and linguistic typology.
Formalizing the computational ATL analogy, per-head cross-attention decomposition, and testing RHM translation asymmetry in the encoder's representational structure are particularly promising avenues.

The \textsc{InterpretCognates} toolkit and the full analysis pipeline---from embedding extraction through statistical testing to figure generation---are released as open-source software to facilitate replication and extension.
We hope that this work illustrates the potential for neural translation models to serve as large-scale computational testbeds for theories of language universals, offering a bridge between the statistical patterns learned from parallel corpora and the conceptual structures that underlie human multilingual cognition.
